\chapter*{ВВЕДЕНИЕ}
\addcontentsline{toc}{chapter}{ВВЕДЕНИЕ}

Параллельные вычисления позволяют увеличить скорость выполнения программы. Конвейерная обработка является приемом, где использование принципов параллельности помогает ускорить обработку данных. Она позволяет на каждой следующей «линии» конвейера использовать данные, полученные с предыдущего этапа.

\textbf{Целью} данной лабораторной работы является описание конвейерной обработки данных.

Необходимо выполнить следующие \textbf{задачи}:
\begin{enumerate}
    \item изучить основы конвейерной обработки;
    \item описать используемые алгоритмы обработки матриц;
    \item выполнить замеры затрат реализаций алгоритмов по процессорному времени;
    \item провести сравнительный анализ алгоритмов.
\end{enumerate}

\chapter{Аналитическая часть}
В этом разделе будут рассмотрена информация, касающаяся основ конвейерной обработки данных.


\section{Конвейерная обработка данных}
Конвейер — организация вычислений, при которой увеличивается количество выполняемых инструкций за единицу времени за счет использования
принципов параллельности \cite{conv}.
Конвейерную обработку можно использовать для совмещения этапов
выполнения разных команд. Производительность при этом возрастает благодаря тому, что одновременно на различных ступенях конвейера выполняются
несколько команд. Такая обработка данных в общем случае основана на разделении подлежащей исполнению функции на более мелкие части, называемые
лентами, и выделении для каждой из них отдельного блока аппаратуры.
Конвейеризация позволяет увеличить пропускную способность процессора (количество команд, завершающихся в единицу времени), но она не
сокращает время выполнения отдельной команды.

\section{Описание алгоритмов}

В качестве примера для операции, подвергающейся конвейерной обработке, будет обрабатываться матрица. Всего будет использовано три ленты,
которые делают следующее:
\begin{enumerate}
    \item матрица упаковываниется по схеме Чанга и Густавсона;
    \item находится определитель матрицы методом миноров;
    \item создается дамп текущей заявки.
\end{enumerate}

% \addcontentsline{toc}{section}{Вывод}
\section*{Вывод}
В этом разделе было рассмотрено понятие конвейрной обработки данных, а также выбраны алгоритмы для обработки матрицы на каждой из трех лент конвейера.
\clearpage
