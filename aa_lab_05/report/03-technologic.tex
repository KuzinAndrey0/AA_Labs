\chapter{Технологическая часть}
В данном разделе будут приведены средства реализации, листинг кода и функциональные тесты.

\section{Средства реализации}

В качестве языка программирования, используемого при написании данной лабораторной работы, был выбран C++ \cite{cpp-lang}. Для замеров времени выбрана библиотека \texttt{<ctime>} \cite{ctime}, позволяющая производить замеры процессорного времени.

В качестве среды для написания кода был выбран \textit{Visual Studio Code} за счет того, что она предоставляет функционал для проектирования, разработки и отладки ПО.

\section{Сведения о модулях программы}

Данная программа разбита на следующие модули:

\begin{itemize}
    \item \texttt{main.cpp}~--- точка входа программы, пользовательское меню;
    \item \texttt{general}~--- модуль с основными конвеерными операциями;
    \item \texttt{csr\_matrix}~--- модуль с реализациями алгоритмов работы над матрицами;
    \item \texttt{measurement}~--- модуль с реализацией функции подсчета затрачиваемого времени.
\end{itemize}



\clearpage

\section{Реализация алгоритмов}

Далее будут представлены реализация для линейного и конвейерного алгоритмов обработки матриц. Также представлена реализация запуска 1, 2 и 3 потоков.


\includelistingpretty
    {general.cpp}
    {c}
    {Реализация конвеерных алгоритмов}
	
\clearpage

\section{Функциональные тесты}

В данном разделе будут представлены функциональные тесты, проверяющие работу алгоритмов сортировок.

В таблице \ref{tbl:func_test} приведены тесты для функций, реализующих алгоритмы сортировок.

\begin{table}[h!]
	\begin{center}
		\caption{\label{tbl:func_test}Тестирование функций}
		\begin{tabular}{|c|c|c|c|}
			\hline
			\textbf{Алгоритм} & \textbf{Кол-во матриц} & \textbf{Размер матриц} & \textbf{Результат} \\ 
			\hline
			Конвейерная & -1  & 10 & Ошибка \\\hline
			Конвейерная & 10  & -1 & Ошибка \\\hline
			Линейная & 10  & 5 & Вывод результата \\\hline
			Конвейерная & 10  & 10 & Вывод результата \\\hline
		\end{tabular}
	\end{center}
\end{table}

% \addcontentsline{toc}{section}{Вывод}
\section*{Вывод}
Были выбраны язык программирования и среда разработки, приведены сведения о модулях программы, листинги алгоритма, проведено функциональное тестирование.