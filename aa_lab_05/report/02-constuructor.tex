\chapter{Конструкторская часть}
В данном разделе будут рассмотрены схемы конвейерной и линейной реализаций алгоритмов обработки матриц.

К программе предъявлен ряд функциональных требований:

\begin{itemize}
    \item наличие интерфейса для выбора действий;
    \item возможность выбора линейной или конвейерной реализации алгоритма;
\end{itemize}

\section{Разработка алгоритмов}

На рисунках \ref{img:lin_conv}--\ref{img:queue3} представлены схемы линейной и конвейерной реализаций алгоритмов обработки матриц, а также схема трех лент для конвейерной
обработки матрицы.

\includeimage
{lin_conv} % Имя файла без расширения (файл должен быть расположен в директории inc/img/)
{f} % Обтекание (без обтекания)
{H} % Положение рисунка (см. figure из пакета float)
{0.5\textwidth} % Ширина рисунка
{Схема алгоритма линейной обработки матрицы} % Подпись рисунка

\includeimage
{conv_conv} % Имя файла без расширения (файл должен быть расположен в директории inc/img/)
{f} % Обтекание (без обтекания)
{H} % Положение рисунка (см. figure из пакета float)
{0.5\textwidth} % Ширина рисунка
{Схема алгоритма конвеерной обработки матрицы} % Подпись рисунка

\includeimage
{queue1} % Имя файла без расширения (файл должен быть расположен в директории inc/img/)
{f} % Обтекание (без обтекания)
{H} % Положение рисунка (см. figure из пакета float)
{0.5\textwidth} % Ширина рисунка
{Схема алгоритма 1 потока обработки матрицы (упаковка матрицы)} % Подпись рисунка

\includeimage
{queue2} % Имя файла без расширения (файл должен быть расположен в директории inc/img/)
{f} % Обтекание (без обтекания)
{H} % Положение рисунка (см. figure из пакета float)
{0.5\textwidth} % Ширина рисунка
{Схема алгоритма 2 потока обработки матрицы (вычисление определителя)} % Подпись рисунка

\includeimage
{queue3} % Имя файла без расширения (файл должен быть расположен в директории inc/img/)
{f} % Обтекание (без обтекания)
{H} % Положение рисунка (см. figure из пакета float)
{0.5\textwidth} % Ширина рисунка
{Схема алгоритма 3 потока обработки матрицы (создание дампа заявки)} % Подпись рисунка


\section*{Вывод}
% \addcontentsline{toc}{section}{Вывод}

В данном разделе были построены схемы алгоритмов требуемых методов
обработки матриц (конвейерного и линейного).

\clearpage