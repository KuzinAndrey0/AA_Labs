\chapter{Исследовательская часть}

\section{Технические характеристики}
Технические характеристики устройства, на котором выполнялись
замеры по времени:

\begin{itemize}
    \item Процессор: Intel Core i7 9750H 2.6 ГГц;
    \item Оперативная память: 16 ГБ;
    \item Операционная система: Kubuntu 22.04.3 LTS x86\_64 Kernel: 6.2.0-36-generic
\end{itemize}

Во время проведения измерений времени ноутбук был подключен к сети электропитания и был нагружен только системными приложениями.

\section{Демонстрация работы программы}

На рисунке \ref{img:img_prog} показан пример работы с программой.


\includeimage
    {img_prog} % Имя файла без расширения (файл должен быть расположен в директории inc/img/)
    {f} % Обтекание (без обтекания)
    {H} % Положение рисунка (см. figure из пакета float)
    {1\textwidth} % Ширина рисунка
    {Демонстрания работы с программой} % Подпись рисунка


\clearpage

\section{Временные характеристики}

Исследование временных характеристик реализованных алгоритмов производилось на следующих запросах:
\begin{enumerate}
    \item случайное слово из словаря;
    \item первое слово из последнего сегмента;
    \item последнее слово из первого сегмента.
\end{enumerate}

\includeimage
    {plotting_data1} % Имя файла без расширения (файл должен быть расположен в директории inc/img/)
    {f} % Обтекание (без обтекания)
    {H} % Положение рисунка (см. figure из пакета float)
    {1.2\textwidth} % Ширина рисунка
    {Результат измерений времени работы (в мс) при разных запросах} % Подпись рисунка


\clearpage

% \begin{figure}[ht!]
% 	\centering
% 	\includesvg[width=1.0\textwidth]{inc/img/plotting_data2.svg}
% 	\caption{Результат измерений времени работы (в мс) алгоритмов при разном кол-ве матриц\label{overflow}}
% 	\label{fig:plotting_data2}
% 	\end{figure}


% \addcontentsline{toc}{section}{Вывод}
\section{Вывод}
В среднем, использование сегментированного словаря занимает меньше времени, по сравнению с обычным словарем. В частности, если ключ ближе к последнему сегменту выигрыш сегментированного словаря становится в 1.5 раза больше. Однако, если поиск стремится к первому сегменту, то выполнение поиска в обычном словаре становится выгоднее.