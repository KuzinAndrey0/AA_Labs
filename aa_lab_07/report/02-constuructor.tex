\chapter{Конструкторская часть}

В этом разделе будут представлено описание используемых типов данных, а также псевдокод алгоритма поиска в словаре.

К программе предъявлен ряд функциональных требований:
\begin{itemize}
	\item иметь интерфейс ввода вопроса;
	\item работа с словарями и строками;
\end{itemize}

\section{Описание используемых типов данных}
При реализации алгоритмов будут использован словрь --- встроенный тип dict в Python:
\section{Разработка алгоритмов}

В листинге\ref{lst:alg1} представлен псевдокод алгоритма поиска в словаре полным перебором.

% \includeimage
% {full_comb} % Имя файла без расширения (файл должен быть расположен в директории inc/img/)
% {f} % Обтекание (без обтекания)
% {H} % Положение рисунка (см. figure из пакета float)
% {1\textwidth} % Ширина рисунка
% {Схема алгоритма поиска в словаре полным перебором} % Подпись рисунка

\begin{lstlisting}[breakatwhitespace=false, label=lst:alg1, caption=Алгоритм поиска в словаре полным перебором]
	input: dictionary dict, string key
   output: value
   
   for each dict.key
   begin
	if dict[key] = key then
	begin
		return dict[key].value
	end
   end
   return "no string found"
   \end{lstlisting}

\section*{Вывод}
% \addcontentsline{toc}{section}{Вывод}

В данном разделе были представлены требования к программе, представлен псевдокод алгоритма поиска значения в словаре.

\clearpage