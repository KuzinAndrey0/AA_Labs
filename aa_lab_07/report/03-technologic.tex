\chapter{Технологическая часть}
В данном разделе будут приведены средства реализации, листинг кода и функциональные тесты.

\section{Средства реализации}

В данном разделе рассмотрены средства реализации, а также представлены листинги реализаций алгоритма расчета термовой частота для всех термов из выборки документов.

В данной работе для реализации был выбран язык программирования $Python$. 
Данный выбор обусловлен соответствием с требованиями выдвинутыми в конструкторской части, а именно в языке программирование $Python$ имеется встроенные типы данных --- словарь и массив и инструменты для поиска подстроки в строке.

\section{Сведения о модулях программы}

Данная программа разбита на следующие модули:

\begin{itemize}
    \item \texttt{main.py}~--- точка входа программы, пользовательское меню;
    \item \texttt{dict.py}~--- модуль с основными операциями над словарями;
\end{itemize}

\clearpage

\section{Реализация алгоритмов}

В листингах \ref{lst:norm_dict.py} -- \ref{lst:segm_dict.py} приведены реализации алгоритмов поиска значения по ключу в обычном словаре и в сегментированном словаре. 


\includelistingpretty
    {norm_dict.py}
    {python}
    {Реализация алгоритма поиска в обычном словаре}

\includelistingpretty
    {segm_dict.py}
    {python}
    {Реализация алгоритма поиска в сегментированном словаре}

% \addcontentsline{toc}{section}{Вывод}
\section*{Вывод}
Были выбраны язык программирования и среда разработки, приведены сведения о модулях программы, листинги алгоритма.