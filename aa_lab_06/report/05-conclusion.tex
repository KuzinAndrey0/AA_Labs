\chapter*{ЗАКЛЮЧЕНИЕ}
\addcontentsline{toc}{chapter}{ЗАКЛЮЧЕНИЕ}

В ходе лабораторной работы поставленная ранее цель была достигнута:
реализованы методы решения задачи коммивояжера на основе муравьиного алгоритма

Исходя из полученных результатов, использование муравьиного алгоритма наиболее эффективно по времени при больших размерах матриц. Так при размере матрицы, равном 2, муравьиный алгоритм медленее алгоритма полного перебора в 153 раза, а при размере матрицы, равном 9, муравьиный алгоритм быстрее алгоритма полного перебора в раз, а при размере в 10 -- уже в 21 раз. Следовательно, при размерах матриц больше 8 следует использовать муравьиный алгоритм, но стоит учитывать, что он не гарантирует оптимального решения, в отличие от метода полного перебора.


В ходе выполнения лабораторной работы были решены следующие задачи:

\begin{itemize}
	\item описана задача коммивояжера;
	\item описаны методы решения задачи коммивояжера;
	\item реализованы выбранные алгоритмы;
	\item проведено функциональное тестирование разработанных алгоритмов;
	\item исследовано время работы реализации алгоритмов. 
\end{itemize}

