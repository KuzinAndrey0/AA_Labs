\documentclass{bmstu}

\bibliography{biblio}
\usepackage{svg}
\usepackage{algpseudocode}
\usepackage{longtable}
\usepackage{listings}

\definecolor{darkgray}{gray}{0.15}

\lstset{
	language=python, % выбор языка для подсветки
	basicstyle=\small\sffamily, % размер и начертание шрифта для подсветки кода
	numbers=left, % где поставить нумерацию строк (слева\справа)
	%numberstyle=, % размер шрифта для номеров строк
	stepnumber=1, % размер шага между двумя номерами строк
	numbersep=5pt, % как далеко отстоят номера строк от подсвечиваемого кода
	frame=single, % рисовать рамку вокруг кода
	tabsize=2, % размер табуляции по умолчанию равен 4 пробелам
	captionpos=t, % позиция заголовка вверху [t] или внизу [b]
	breaklines=true,
	breakatwhitespace=true, % переносить строки только если есть пробел
	backgroundcolor=\color{white},
	keywordstyle=\color{blue}
}

\begin{document}

\setlist[enumerate]{label=\arabic*)}
\setlist[itemize]{label=---}

\makereporttitle{Информатика и системы управления}
{Программное обеспечение ЭВМ и информационные технологии}
{лабораторной работе № 6}
{Анализ Алгоритмов}
{Методы решения задачи коммивояжёра}
{}{А. А. Кузин/ИУ7-51Б}{Л. Л. Волкова}


\renewcommand{\contentsname}{Содержание}
\tableofcontents
\setcounter{page}{2}
\chapter*{Введение}
\addcontentsline{toc}{chapter}{Введение}

\textbf{Сортировка} — это алгоритм для упорядочивания элементов в списке. В случае, когда элемент в списке имеет несколько полей, поле, служащее критерием порядка, называется ключом сортировки. На практике в качестве ключа часто выступает число, а в остальных полях хранятся какие-либо данные, никак не влияющие на работу алгоритма.

\textbf{Целью} данной лабораторной работы является анализ алгоритмов сортировок.

Необходимо выполнить следующие \textbf{задачи}:
\begin{enumerate}
    \item Ознакомиться со следующими алгоритмами сортировок:
    \begin{itemize}    
        \item Плавная сортировка (smoothsort);
        \item Сортировка перемешиванием;
        \item Сортировка Шелла.
    \end{itemize}
    \item Реализовать алгоритмы сортировок;
    \item Выполнить замеры затрат реализаций алгоритмов по памяти;
    \item Выполнить замеры затрат реализаций алгоритмов по процессорному времени;
    \item Провести сравнительных анализ алгоритмов.
\end{enumerate}

\chapter{Аналитическая часть}

\section{Плавная сортировка (smoothsort)}
Плавная сортировка (англ. Smooth sort) — модификация сортировки кучей, разработанный Э. Дейкстрой. Как и пирамидальная сортировка, в худшем случае работает за время $\Theta(N\log{N})$. Преимущество плавной сортировки в том, что её время работы приближается к $O(N)$, если входные данные частично отсортированы, в то время как у сортировки кучей время работы не зависит от состояния входных данных.
\section{Сортировка перемешиванием}
Сортировка перемешиванием \cite{shaker} -- это разновидность сортировки пузырьком. Отличие в том, что данная сортировка в рамках одной итерации проходит по массиву в обоих направлениях (слева направо и справа налево), тогда как сортировка пузырьком -- только в одном направлении (слева направо).

Общие идеи алгоритма:
\begin{itemize}
	\item обход массива слева направо, аналогично пузырьковой -- сравнение соседних элементов, меняя их местами, если левое значение больше правого;
	\item обход массива в обратном направлении (справа налево), начиная с элемента, который находится перед последним отсортированным, то есть на этом этапе элементы также сравниваются между собой и меняются местами, чтобы наименьшее значение всегда было слева.
\end{itemize}
\section{Сортировка Шелла}
Метод предложен в 1959 году и назван по имени автора метода Дональда Шелла (Donald Shell). 

Сортировка Шелла \cite{book_lipachev, book_sort_algorithms, book_knut} (англ. Shell Sort) является улучшением сортировки вставками. Часто еще называемый "сортировка вставками с уменьшением расстояния". Основная идея этого метода заключается в том, чтобы в начале устранить массовый беспорядок в массиве, сравнивая далеко
отстоящие друг от друга элементы. Постепенно интервал между сравниваемыми элементами уменьшается до единицы. Это означает, что на поздних стадиях сортировка сводится просто к перестановкам соседних элементов (если, конечно, такие перестановки являются необходимыми).

Пусть $d$ - интервал между сравниваемыми элементами. Первоначально используемая Шеллом последовательность длин промежутков: $d_1 = \frac{N}{2}, d_{i} = \frac{d_{i - 1}}{2}, … d_k = 1$. Процесс завершается обычной сортировкой вставками получившегося списка.
\addcontentsline{toc}{section}{Вывод}
\section*{Вывод}
В данном разделе были рассмотрены следующие алгоритмы сортировок: 
\begin{itemize}    
    \item Плавная сортировка;
    \item Сортировка перемешиванием;
    \item Сортировка Шелла.
\end{itemize}
\clearpage

\chapter{Конструкторская часть}

В этом разделе будут представлено описание используемых типов данных, а также псевдокод алгоритма поиска в словаре.

К программе предъявлен ряд функциональных требований:
\begin{itemize}
	\item иметь интерфейс ввода вопроса;
	\item работа с словарями и строками;
\end{itemize}

\section{Описание используемых типов данных}
При реализации алгоритмов будут использован словрь --- встроенный тип dict в Python:
\section{Разработка алгоритмов}

В листинге\ref{lst:alg1} представлен псевдокод алгоритма поиска в словаре полным перебором.

% \includeimage
% {full_comb} % Имя файла без расширения (файл должен быть расположен в директории inc/img/)
% {f} % Обтекание (без обтекания)
% {H} % Положение рисунка (см. figure из пакета float)
% {1\textwidth} % Ширина рисунка
% {Схема алгоритма поиска в словаре полным перебором} % Подпись рисунка

\begin{lstlisting}[breakatwhitespace=false, label=lst:alg1, caption=Алгоритм поиска в словаре полным перебором]
	input: dictionary dict, string key
   output: value
   
   for each dict.key
   begin
	if dict[key] = key then
	begin
		return dict[key].value
	end
   end
   return "no string found"
   \end{lstlisting}

\section*{Вывод}
% \addcontentsline{toc}{section}{Вывод}

В данном разделе были представлены требования к программе, представлен псевдокод алгоритма поиска значения в словаре.

\clearpage
\chapter{Технологическая часть}
В данном разделе будут приведены средства реализации, листинг кода и функциональные тесты.

\section{Средства реализации}

В качестве языка программирования, используемого при написании данной лабораторной работы, был выбран C++ \cite{cpp-lang}. Для замеров времени выбрана библиотека \texttt{<ctime>} \cite{ctime}, позволяющая производить замеры процессорного времени.

В качестве среды для написания кода был выбран \textit{Visual Studio Code} за счет того, что она предоставляет функционал для проектирования, разработки и отладки ПО.

\section{Сведения о модулях программы}

Данная программа разбита на следующие модули:

\begin{itemize}
    \item \texttt{main.cpp}~--- точка входа программы, пользовательское меню;
    \item \texttt{general}~--- модуль с основными конвеерными операциями;
    \item \texttt{csr\_matrix}~--- модуль с реализациями алгоритмов работы над матрицами;
    \item \texttt{measurement}~--- модуль с реализацией функции подсчета затрачиваемого времени.
\end{itemize}



\clearpage

\section{Реализация алгоритмов}

Далее будут представлены реализация для линейного и конвейерного алгоритмов обработки матриц. Также представлена реализация запуска 1, 2 и 3 потоков.


\includelistingpretty
    {general.cpp}
    {c}
    {Реализация конвеерных алгоритмов}
	
\clearpage

\section{Функциональные тесты}

В данном разделе будут представлены функциональные тесты, проверяющие работу алгоритмов сортировок.

В таблице \ref{tbl:func_test} приведены тесты для функций, реализующих алгоритмы сортировок.

\begin{table}[h!]
	\begin{center}
		\caption{\label{tbl:func_test}Тестирование функций}
		\begin{tabular}{|c|c|c|c|}
			\hline
			\textbf{Алгоритм} & \textbf{Кол-во матриц} & \textbf{Размер матриц} & \textbf{Результат} \\ 
			\hline
			Конвейерная & -1  & 10 & Ошибка \\\hline
			Конвейерная & 10  & -1 & Ошибка \\\hline
			Линейная & 10  & 5 & Вывод результата \\\hline
			Конвейерная & 10  & 10 & Вывод результата \\\hline
		\end{tabular}
	\end{center}
\end{table}

% \addcontentsline{toc}{section}{Вывод}
\section*{Вывод}
Были выбраны язык программирования и среда разработки, приведены сведения о модулях программы, листинги алгоритма, проведено функциональное тестирование.
\chapter{Исследовательская часть}

\section{Технические характеристики}
Технические характеристики устройства, на котором выполнялись
замеры по времени:

\begin{itemize}
    \item Процессор: Intel Core i7 9750H 2.6 ГГц;
    \item Оперативная память: 16 ГБ;
    \item Операционная система: Kubuntu 22.04.3 LTS x86\_64 Kernel: 6.2.0-36-generic
\end{itemize}

Во время проведения измерений времени ноутбук был подключен к сети электропитания и был нагружен только системными приложениями.

\section{Демонстрация работы программы}

На рисунке \ref{fig:img_prog} показан пример работы с программой.


\begin{figure}[ht!]
	\centering
	\includegraphics[width=170mm]{img/img_prog.png}
	\caption{Демонстрация работы программы.\label{overflow}}
	\label{fig:img_prog}
	\end{figure}


\clearpage

\section{Временные характеристики}


Исследование временных характеристик реализованных алгоритмов производилось на массивах размером 1 -- 8 с шагом 1.

\begin{figure}[ht!]
	\centering
	\includesvg[width=1.0\textwidth]{inc/img/plotting_data1.svg}
	\caption{Результат измерений времени работы (в мс) алгоритмов при разных размерах матриц\label{overflow}}
	\label{fig:plotting_data1}
	\end{figure}
\clearpage

\begin{figure}[ht!]
	\centering
	\includesvg[width=1.0\textwidth]{inc/img/plotting_data2.svg}
	\caption{Результат измерений времени работы (в мс) алгоритмов при разном кол-ве матриц\label{overflow}}
	\label{fig:plotting_data2}
	\end{figure}


% \addcontentsline{toc}{section}{Вывод}
\section{Вывод}

При разных размерах матриц, в связи с разной сложностью заявок, при больших размерах матриц возникает проблема горлышка бутылки. Из-за экспоненциального роста второй заявки разница между конвеерным и линейным алгоритмом стремится к нулю.

При одинаковых размерах матриц, конвеерная обработка заявок занимает меньше времени, чем линейная обработка.
\chapter*{ЗАКЛЮЧЕНИЕ}
\addcontentsline{toc}{chapter}{ЗАКЛЮЧЕНИЕ}

В ходе лабораторной работы поставленная ранее цель была достигнута:
были изучены принципы конвейерной обработки данных на примере работы
с матрицами. 

В ходе выполнения лабораторной работы были решены следующие задачи:

\begin{enumerate}
    \item изучены основы конвейерной обработки;
    \item описаны используемые алгоритмы обработки матриц;
    \item выполнены замеры затрат реализаций алгоритмов по процессорному времени;
    \item проведены сравнительный анализ алгоритмов.
\end{enumerate}


\makebibliography

\end{document}