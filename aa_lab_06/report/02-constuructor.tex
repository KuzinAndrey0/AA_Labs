\chapter{Конструкторская часть}
В данном разделе будут рассмотрены схемы конвейерной и линейной реализаций алгоритмов обработки матриц.

К программе предъявлен ряд функциональных требований:

\begin{itemize}
    \item наличие интерфейса для выбора действий;
    \item возможность выбора линейной или конвейерной реализации алгоритма;
\end{itemize}

\section{Описание используемых типов данных}
При реализации алгоритмов будут использованы следующие типы дан-
ных:
\begin{itemize}[label=---]
	\item размер матрицы смежности --- целое число;
	\item имя файла --- строка;
	\item коэффициенты $\alpha, \beta$, \textit{k\_evaporation} --- действительные числа;
	\item матрица смежности --- матрица целых чисел.
\end{itemize}


\section{Разработка алгоритмов}

 
На рисунке \ref{img:perebor} представлена схема алгоритма полного перебора путей, а на рисунках \ref{img:angs} схема муравьиного алгоритма поиска путей. Также на рисунках \ref{img:find_pos}--\ref{img:update_ph} представлены схемы вспомогательных функций для муравьиного алгоритма.

\includeimage
{perebor} % Имя файла без расширения (файл должен быть расположен в директории inc/img/)
{f} % Обтекание (без обтекания)
{H} % Положение рисунка (см. figure из пакета float)
{0.5\textwidth} % Ширина рисунка
{Схема алгоритма полного перебора путей} % Подпись рисунка

\includeimage
{angs} % Имя файла без расширения (файл должен быть расположен в директории inc/img/)
{f} % Обтекание (без обтекания)
{H} % Положение рисунка (см. figure из пакета float)
{0.5\textwidth} % Ширина рисунка
{Схема муравьиного алгоритма} % Подпись рисунка

\includeimage
{find_pos} % Имя файла без расширения (файл должен быть расположен в директории inc/img/)
{f} % Обтекание (без обтекания)
{H} % Положение рисунка (см. figure из пакета float)
{0.5\textwidth} % Ширина рисунка
{Схема нахождения массива вероятностей переходов} % Подпись рисунка

\includeimage
{update_ph} % Имя файла без расширения (файл должен быть расположен в директории inc/img/)
{f} % Обтекание (без обтекания)
{H} % Положение рисунка (см. figure из пакета float)
{0.5\textwidth} % Ширина рисунка
{Схема обновления матрицы феромонов} % Подпись рисунка



\section*{Вывод}
% \addcontentsline{toc}{section}{Вывод}

В данном разделе были построены схемы алгоритмов требуемых методов
решения задачи коммивояжёра.

\clearpage