\chapter{Технологическая часть}
В данном разделе будут приведены средства реализации, листинг кода и функциональные тесты.

\section{Средства реализации}

В качестве языка программирования, используемого при написании данной лабораторной работы, был выбран C++ \cite{cpp-lang}, так как в нем имеется контейнер \texttt{std::string}, представляющий собой массив символов \texttt{char}, и библиотека \texttt{<ctime>} \cite{ctime}, позволяющая производить замеры процессорного времени.

В качестве среды для написания кода был выбран \textit{Visual Studio Code} за счет того, что она предоставляет функционал для проектирования, разработки и отладки ПО.

\section{Сведения о модулях программы}

Данная программа разбита на следующие модули:

\begin{itemize}
    \item \texttt{main.cpp}~--- точка входа программы, пользовательское меню;
    \item \texttt{general}~--- модуль с определением матрицы;
    \item \texttt{algo}~--- модуль с реализациями алгоритмов умножения матриц;
    \item \texttt{measurement}~--- модуль с реализаций функции подсчета затрачиваемого времени.
\end{itemize}

\clearpage

\section{Реализация алгоритмов}

Далее будут представлены реализации следующих алгоритмов умножения матриц:
\begin{itemize}
    \item Листинг \ref{lst:smooth.cpp}~--- Плавная сортировка;
    \item Листинг \ref{lst:shaker.cpp}~--- Сортировка перемешиванием;
    \item Листинг \ref{lst:shell.cpp}~--- Сортировка Шелла.
\end{itemize}



\includelistingpretty
    {smooth.cpp}
    {c}
    {Плавная сортировка}
	
\includelistingpretty
    {shaker.cpp}
    {c}
    {Сортировка перемешиванием}	

\includelistingpretty
    {shell.cpp}
    {c}
    {Сортировка Шелла}	

\clearpage

\section{Функциональные тесты}

В данном разделе будут представлены функциональные тесты, проверяющие работу алгоритмов сортировок.

В таблице \ref{tbl:func_test} приведены тесты для функций, реализующих алгоритмы сортировок.

\begin{table}[h!]
	\begin{center}
		\caption{\label{tbl:func_test}Тестирование функций}
		\begin{tabular}{|c|c|c|}
			\hline
			\textbf{Входной массив} & \textbf{Результат} & \textbf{Ожидаемый результат} \\ 
			\hline
			$[15, 25, 35, 45]$ & $[15, 25, 35, 45]$  & $[15, 25, 35, 45]$\\\hline
			$[55, 45, 35, 25]$  & $[25, 35, 45, 55]$ & $[25, 35, 45, 55]$\\\hline
			$[-10, -20, -30, -25]$  & $[-30, -25, -20, -10]$  & $[-30, -25, -20, -10]$\\\hline
			$[40, -10, -30, 75]$  & $[-30, -10, 40, 75]$  & $[-30, -10, 40, 75]$\\\hline
			$[100]$  & $[100]$  & $[100]$\\\hline
			$[-20]$  & $[-20]$  & $[-20]$\\\hline
			$[1.1, 2.2, 3.3, 4.4]$  & $[1.1, 2.2, 3.3, 4.4]$  & $[1.1, 2.2, 3.3, 4.4]$\\\hline
			$[1.1, -2.2, 3.3, -4.4]$  & $[-4.4, -2.2, 1.1, 3.3]$  &  $[-4.4, -2.2, 1.1, 3.3]$\\\hline
			$[-1.1, -2.2, -3.3, -4.4]$  & $[-4.4, 3.3, -2.2, -1.1]$  &  $[-4.4, -3.3, -2.2, -1.1]$\\\hline
			$[10, 10]$  & $[10, 10]$  & $[10, 10]$ \\\hline
		\end{tabular}
	\end{center}
\end{table}

\clearpage

\addcontentsline{toc}{section}{Вывод}
\section*{Вывод}

Были реализованы алгоритм плавной сортировки, алгоритм сортировки перемешиванием, алгоритм сортировки Шелла.
Проведено тестирование реализованных алгортимов.