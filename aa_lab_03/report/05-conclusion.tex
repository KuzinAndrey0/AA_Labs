\chapter*{Заключение}
\addcontentsline{toc}{chapter}{Заключение}

По времени выполнения, алгоритм сортировки Шелла является самым эффективным, имея время выполнения быстрее других рассмотренных алгоритмов при любых типах массов.

По затрачиваемой памяти, алгоритм сортировки перемешиванием является самым эффективным, требуя наименьший объем затрачиваемой памяти.

Самыми эффективными по трудоемкости при неотсортированном массиве являются сортировки перемешиваем и Шелла, имеющие одинаковую асимптотическую сложность.

В ходе выполнения лабораторной работы были решены следующие задачи:

\begin{enumerate}
    \item Были ознакомлены со следующими алгоритмами сортировок:
    \begin{itemize}    
        \item Плавная сортировка (smoothsort);
        \item Сортировка перемешиванием;
        \item Сортировка Шелла.
    \end{itemize}
    \item Были реализованы алгоритмы сортировок;
    \item Выполнены замеры затрат реализаций алгоритмов по памяти;
    \item Выполнены замеры затрат реализаций алгоритмов по процессорному времени;
    \item Проведен сравнительный анализ алгоритмов.
\end{enumerate}