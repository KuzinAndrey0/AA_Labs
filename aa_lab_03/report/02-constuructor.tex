\chapter{Конструкторская часть}

В данном разделе будут приведены псевдокоды алгоритмов сортировок, описание используемых типов данных и структуры программного обеспечения.

К программе предъявлен ряд функциональных требований:

\begin{itemize}
    \item наличие интерфейса для выбора действий;
    \item возможность ввода чисел, используя ввод с клавиатуры;
    \item возможность произвести замеры процессорного времени работы реализованных алгоритмов сортировок.
\end{itemize}

\section{Разработка алгоритмов}

В листинге \ref{lst:shaker.pas} представлен псевдокод сортировки перемешиванием.

В листинге \ref{lst:shell.pas} представлен псевдокод сортировки Шелла.

В листинге \ref{lst:smooth.pas} представлен псевдокод плавной сортировки.



\includelistingpretty
    {shaker.pas}
    {pascal}
    {псевдокод сортировки перемешиванием}


\includelistingpretty
    {shell.pas}
    {pascal}
    {псевдокод сортировки Шелла}

\includelistingpretty
    {smooth.pas}
    {pascal}
    {псевдокод плавной сортировки}



\section{Описание используемых типов данных}

При реализации алгоритмов будет использован \textit{массив}~--- упорядоченный набор элементов \texttt{целочисленного типа}.

\section{Модель вычислений для проведения оценки трудоемкости алгоритмов}

Для последующего вычисления трудоемкости необходимо ввести модель вычислений:

\begin{enumerate}
	\item операции из списка \ref{eq:operations1} имеют трудоемкость \textbf{1};
	\begin{equation}
		\label{eq:operations1}
		\begin{gathered}
			+, -, =, +=, -=, ==, !=, <, >, <=, >=, [], \\ ++, --, \&\&, >>, <<, ||, \&, |
		\end{gathered}
	\end{equation}
	\item операции из списка \ref{eq:operations2} имеют трудоемкость \textbf{2};
	\begin{equation}
		\label{eq:operations2}
		*, /, \%, *=, /=, \%=
	\end{equation}
	\item трудоемкость условного оператора \texttt{if условие then A else B} рассчитывается как \ref{eq:if};
	\begin{equation}
		\label{eq:if}
		f_{if} = f_{\text{условия}} + 
		\begin{cases}
			f_{A}, & \text{в случае выполнеия условия,}\\
			f_{B}, & \text{иначе}.
		\end{cases}
	\end{equation}
	\item трудоемкость цикла рассчитывается как \ref{eq:for}
	\begin{equation}
		\label{eq:for}
		\begin{gathered}
			f_{for} = f_{\text{инициализация}} + f_{\text{сравнения}} + M_{\text{итераций}} \cdot (f_{\text{тело}} +\\
			+ f_{\text{инкремент}} + f_{\text{сравнения}});
		\end{gathered}
	\end{equation}
	\item трудоемкость вызова функции равна 0.
\end{enumerate}

\section{Трудоемкость алгоритмов}
В следующих частях будут приведены рассчеты трудоемкостей алгоритмов сортировок.

\subsection{Плавная сортировка (smoothsort)}
На первом этапе перебирается N элементов, добавляя его в уже имеющиеся слева кучи. Добавление в кучу обходится в
\begin{equation}
  f_{elem} = O(1)
\end{equation}
  затем для кучи нужно сделать просейку.

В неупорядоченных данных просейка для каждого добавления обходится в:
\begin{equation}
f_{unsort} = O(log(N))
\end{equation}
так как из-за случайных чисел просейке приходится проходить уровни дерева часто до самого низа.

Поэтому, на первом этапе наилучшая сложность по времени:
для упорядоченных чисел:
\begin{equation}
f_{sort} = O(N)
\end{equation} 
для случайных чисел:
\begin{equation}
f_{unsort} = O(N \cdot log(N))
\end{equation} 

Для второго этапа ситуация аналогичная. При обмене очередного максимума опять необходимо просеять кучу, в корне которой он находился.

На втором этапе наилучшая сложность по времени такая же как и на первом:
для упорядоченных чисел:
\begin{equation}
f_{sort} = O(N)
\end{equation} 
для случайных чисел:
\begin{equation}
f_{unsort} = O(N \cdot log(N))
\end{equation} 

Складывая временные сложности для первого и второго этапа:
для упорядоченных чисел:
\begin{equation}
f_{best} = O(2 \cdot N) = O(N)
\end{equation}  
В общем, худшая и средняя временная сложность для плавной сортировки:
\begin{equation}
f_{worst} = O(2 \cdot N \cdot log(N)) = O(N \cdot log(N))
\end{equation} 



\subsection{Сортировка перемешиванием}
\begin{itemize}
	\item Трудоёмкость сравнения внешнего цикла $while$, которая вычисляется по формуле (\ref{for:bubble_outer}).
	\begin{equation}
		\label{for:bubble_outer}
		f_{outer} = 1 + 2 \cdot (N - 1).
	\end{equation}
	\item Суммарная трудоёмкость внутренних циклов, количество итераций которых меняется в промежутке $[1..N-1]$, которая вычисляется по формуле (\ref{for:bubble_inner}).
	\begin{equation}
		\label{for:bubble_inner}
		f_{inner} = 5(N - 1) + \frac{2 \cdot (N - 1)}{2} \cdot (3 + f_{if}).
	\end{equation}
	\item Трудоёмкость условия во внутреннем цикле, которая вычисляется по формуле (\ref{for:bubble_if}).
	\begin{equation}
		\label{for:bubble_if}
		f_{if} = 4 + \begin{cases}
			0, & \text{л.с.}\\
			9, & \text{х.с.}\\
		\end{cases}
	\end{equation}
\end{itemize}

Трудоёмкость в \textbf{лучшем} случае (\ref{for:bubble_best}):
\begin{equation}
	\label{for:bubble_best}
	f_{best} = -3 + \frac{3}{2} N + \approx \frac{3}{2} N = O(N).
\end{equation}

Трудоёмкость в \textbf{худшем} случае (\ref{for:bubble_worst}):
\begin{equation}
	\label{for:bubble_worst}
	f_{worst} = -3 - 8N + 8N^2 \approx 8N^2 = O(N^2).
\end{equation}
\subsection{Сортировка Шелла}
Трудоемкость в лучшем случае при отсортированном массиве, когда ничего не обменивается, но все же данные рассматриваются. Выведена в \newline формуле (\ref{сomplexity:shell_best}).
\begin{equation}
	\label{сomplexity:shell_best}
	\begin{gathered}
		f_{best} = 3 + 4 + \frac{N}{4} \cdot (3 + 2 + log(N) \cdot (2 + 4 + 4)) = \\
		= 7 + \frac{5N}{4} + \frac{5N \cdot log(N)}{2} = O(N \cdot log(N))
	\end{gathered}
\end{equation}

Трудоемкость в худшем случае при отсортированном массиве в обратном порядке. Выведена в формуле (\ref{сomplexity:shell_worst}).
\begin{equation}
	\label{сomplexity:shell_worst}
	\begin{gathered}
		f_{worst} = 7 + \frac{N}{4} \cdot (5 + log(N) \cdot (10 + log(N) \cdot (6 + 4))) =\\
		= 7 + \frac{5N}{4} + \frac{5N \cdot log(N)}{2} + \frac{5N \cdot log^2(N)}{2} = O(N \cdot log^2(N))
	\end{gathered}
\end{equation}

\section*{Вывод}
\addcontentsline{toc}{section}{Вывод}

В данном разделе на основе теоретических данных были перечислены требования к ПО, сделаны псевдокоды реализуемых алгоритмов и представлена модель вычислений на основе данных, полученных на этапе анализа.

При отсортированном массиве асимптотическая сложность алгоритмов плавной сортировки и сортировки перемешиванием стремится к $O(N)$, в то время как сортировка шелла к $O(N)$.

Самыми эффективными по трудоемкости при неотсортированном массиве являются сортировки перемешиваем и Шелла, имеющие одинаковую асимптотическую сложность -  $O(N \cdot log(N))$


\clearpage