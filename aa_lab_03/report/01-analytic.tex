\chapter*{Введение}
\addcontentsline{toc}{chapter}{Введение}

\textbf{Сортировка} — это алгоритм для упорядочивания элементов в списке. В случае, когда элемент в списке имеет несколько полей, поле, служащее критерием порядка, называется ключом сортировки. На практике в качестве ключа часто выступает число, а в остальных полях хранятся какие-либо данные, никак не влияющие на работу алгоритма.

\textbf{Целью} данной лабораторной работы является анализ алгоритмов сортировок.

Необходимо выполнить следующие \textbf{задачи}:
\begin{enumerate}
    \item Ознакомиться со следующими алгоритмами сортировок:
    \begin{itemize}    
        \item Плавная сортировка (smoothsort);
        \item Сортировка перемешиванием;
        \item Сортировка Шелла.
    \end{itemize}
    \item Реализовать алгоритмы сортировок;
    \item Выполнить замеры затрат реализаций алгоритмов по памяти;
    \item Выполнить замеры затрат реализаций алгоритмов по процессорному времени;
    \item Провести сравнительных анализ алгоритмов.
\end{enumerate}

\chapter{Аналитическая часть}

\section{Плавная сортировка (smoothsort)}
Плавная сортировка (англ. Smooth sort) — модификация сортировки кучей, разработанный Э. Дейкстрой. Как и пирамидальная сортировка, в худшем случае работает за время $\Theta(N\log{N})$. Преимущество плавной сортировки в том, что её время работы приближается к $O(N)$, если входные данные частично отсортированы, в то время как у сортировки кучей время работы не зависит от состояния входных данных.
\section{Сортировка перемешиванием}
Сортировка перемешиванием \cite{shaker} -- это разновидность сортировки пузырьком. Отличие в том, что данная сортировка в рамках одной итерации проходит по массиву в обоих направлениях (слева направо и справа налево), тогда как сортировка пузырьком -- только в одном направлении (слева направо).

Общие идеи алгоритма:
\begin{itemize}
	\item обход массива слева направо, аналогично пузырьковой -- сравнение соседних элементов, меняя их местами, если левое значение больше правого;
	\item обход массива в обратном направлении (справа налево), начиная с элемента, который находится перед последним отсортированным, то есть на этом этапе элементы также сравниваются между собой и меняются местами, чтобы наименьшее значение всегда было слева.
\end{itemize}
\section{Сортировка Шелла}
Метод предложен в 1959 году и назван по имени автора метода Дональда Шелла (Donald Shell). 

Сортировка Шелла \cite{book_lipachev, book_sort_algorithms, book_knut} (англ. Shell Sort) является улучшением сортировки вставками. Часто еще называемый "сортировка вставками с уменьшением расстояния". Основная идея этого метода заключается в том, чтобы в начале устранить массовый беспорядок в массиве, сравнивая далеко
отстоящие друг от друга элементы. Постепенно интервал между сравниваемыми элементами уменьшается до единицы. Это означает, что на поздних стадиях сортировка сводится просто к перестановкам соседних элементов (если, конечно, такие перестановки являются необходимыми).

Пусть $d$ - интервал между сравниваемыми элементами. Первоначально используемая Шеллом последовательность длин промежутков: $d_1 = \frac{N}{2}, d_{i} = \frac{d_{i - 1}}{2}, … d_k = 1$. Процесс завершается обычной сортировкой вставками получившегося списка.
\addcontentsline{toc}{section}{Вывод}
\section*{Вывод}
В данном разделе были рассмотрены следующие алгоритмы сортировок: 
\begin{itemize}    
    \item Плавная сортировка;
    \item Сортировка перемешиванием;
    \item Сортировка Шелла.
\end{itemize}
\clearpage
